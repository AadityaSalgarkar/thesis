%!TEX root = ../Central_compile.tex
%%%%%%%%%%%%%%%%%%%%%%%%%%%%%%%%%%%%%%%%%%%%%

%%%%%%%%%%%%%%%%%%%%%%%%%%%%%%%%%%%%%%%%%%%%%%%%%%%%%%%%%%%%%%%%%%%%%%%%%%%%%%%%%%%%%%%%%%
\section{Flat space limit}
\label{sec:flat_space_limit}
%%%%%%%%%%%%%%%%%%%%%%%%%%%%%%%%%%%%%%%%%%%%%%%%%%%%%%%%%%%%%%%%%%%%%%%%%%%%%%%%%%%%%%%%%%

Having fixed the general form of the impact parameter representation of the one-loop correlator from first principles in section \ref{sec:ads},
we now want to fix part of the dynamical data by taking the flat space limit,
which relates it to the known flat space amplitudes.
The prescription to achieve this was discovered in
\cite{Cornalba:2007fs}, where it was applied to scalar tree-level amplitudes.
This limit is taken by sending the AdS radius $R$ to infinity while scaling the relevant quantities in order to match them to flat space quantities in a sensible way.
The dimensionless quantities $S$ and $L$ are sent to dimensionless combinations of $R$ with the flat space center of mass energy $s$ and impact parameter $b$ as
\beq
 S = \frac{R^2 s}{4}\,, \qquad  L=\frac{b}{R}\,.
\eeq
Note that $L$ is the AdS impact parameter, as it describes the geodesic distance on $H_{d-1}$ between the impact points in transverse space.
If we impose the identification of Casimir eigenvalues
\beq
\De (\De-d) = R^2 m^2\,,
\label{eq:casimirdeltam}
\eeq
for the states on the leading Regge trajectory and take this equation off-shell,
it becomes
\beq
\nu^2 + \left( \frac{d}{2} \right)^2 =  R^2 q^2\,,
\eeq
so for large $R$ we further impose
\beq
\nu^2 = R^2 q^2\,.
\eeq
Our expressions in AdS are integrals in $\nu$, while in flat space we have vector integrals in $q$,
where we recall that $q$ is a vector in the transverse space $\mathbb{R}^{D-2}$.
In order to compare the expressions, it is instructive to do the flat space angular integrals and keep only the integral over the modulus $|q|$.
In this way, the exponential is replaced by the harmonic function $\omega_q(b)$ according to 
\beq
\int\limits_{\mathbb{R}^{D-2}} \frac{d q}{(2\pi)^{D-2}} \, e^{i b \cdot q}
= 2 \int\limits_0^{\infty} d|q| \, \omega_q(b)\,,
\label{eq:exp_to_harmonic}
\eeq
so that \cite{Costa:2014kfa}
	\bea
\omega_{q} (b) = q \int\limits_{\mathbb{R}^{D-2}} \frac{d p}{(2\pi)^{D-2}} \, e^{i b \cdot p}\, \delta(p^2 - q^2) = \frac{1}{2 (2\pi)^{\frac{D-2}{2}}}\frac{|q|^\frac{D-2}{2}}{|b|^{\frac{D-4}{2}}} J_{\frac{D-4}{2}}(|q| |b|)\,,
	\eea{eq:harmonic_flat}
where $J$ denotes the Bessel $J$-function and we recall that $\omega_{q} (b)$ only depends on the modulus of the vectors $q$ and $b$.
One can check that the flat space limit of the $H_{d-1}$ harmonic function \eqref{eq:Omega} yields the flat space harmonic function
\beq
R^{3-D} \Omega_{i \nu} (L) \to \omega_{q} (b)\,, \qquad \nu \geq 0\,.
\label{eq:Omega_fsl}
\eeq
For even $d$ this can be checked directly, while for general $d$ it is convenient to use an integral representation for the hypergeometric function which under the limit is related to an integral representation for the Bessel function \cite{Carmi:2018qzm}.
For even $d$ the relation is also valid for $\nu < 0$.

In the context of string theory we further have the dimensionless coupling $\lambda$ which is expressed in terms of $\alpha'$ and $R$ as
\beq
\sqrt{\l} = \frac{R^2}{\a'}\,,
\eeq
meaning we can also express $S$ as
\beq
S= \frac{\sqrt{\lambda}\,  \alpha's}{4}\,.
\label{eq:Stos}
\eeq
To summarize, the flat space limit is taken by
sending the AdS radius $R$ to infinity while replacing
\beq
S = \frac{\sqrt{\lambda} \alpha' s}{4}, \quad L=\frac{b}{R}, \quad	\nu^2 = R^2 q^2, \quad \nu_1^2 = R^2 q_1^2, \quad  \nu_2^2 = R^2 q_2^2, \quad \sqrt{\l} = \frac{R^2}{\a'}  \,,
		\label{eq:flat_space_limit}
\eeq
and impact parameter representations can be compared by using \eqref{eq:Omega_fsl}.
We can also use these relations to relate $\Delta_{gap}$ to $\lambda$ taking as reference a string state of mass $m^2=4/\alpha'$, therefore
\beq
\Delta^2_{\text{gap}}=\frac{4 R^2}{\alpha'} = 4 \sqrt{\lambda}\,.
\eeq

%%%%%%%%%%%%%%%%%%%%%%%%%%%%%%%%%%%%%%%%%%%%%%%%%%%%%%%%%%%%%%%%%%%%%%%%%%%%%%%%%
\subsection{Matching in impact parameter space}
\label{sec:matching_impact_parameter}
%%%%%%%%%%%%%%%%%%%%%%%%%%%%%%%%%%%%%%%%%%%%%%%%%%%%%%%%%%%%%%%%%%%%%%%%%%%%%%%%%
Let us now see what we can learn when we apply the flat space limit to the impact parameter representations studied in section \ref{sec:ads}. We begin with the tree-level correlator of four scalars for which the limit was originally imposed in \cite{Cornalba:2007fs}.
The flat space limit of the AdS result $\cB$ in \eqref{eq:B} should match the flat space impact parameter representation $i \de$ from \eqref{eq:impact_flat} of the amplitude \eqref{eq:A_tree_regge}\footnote{The relative factor $i$ in $\cB \to i \de$ can be determined by matching the exponents in the eikonal approximation for $\lambda \to \infty$.}
\beq
\cB_{\text{tree}}(p,\pb) = 4 \pi i \int\limits_{0}^{\oo} d\nu \, \b(\nu)\,S^{j(\nu)-1}\, \Omega_{i\nu} (L)
\ \to \ 
i\de_{\text{tree}} (s,b) = 2i \int\limits_0^{\oo} d|q|\, \b(t) \left(\frac{\alpha's}{4}\right)^{j(t)-1} \omega_q(b) \,.
\label{eq:flat_space_limit_dilatons}
\eeq
Here and below we do not always write the overall factors of $R$ as in \eqref{eq:Omega_fsl}, but they do work out correctly when including the expansion parameters from
\eqref{eq:loop_expansion} and \eqref{eq:amplitude_loop_expansion} and using the relation
\beq
\frac{1}{N^2} = \frac{1}{R^{D-2}} \frac{2 G_N}{\pi}\,.
\eeq
From \eqref{eq:Omega_fsl} and \eqref{eq:Stos} we see that
this does indeed match, provided  the flat space limit of the AdS Regge trajectory and spectral function are sent to the flat space Regge trajectory and Pomeron propagator%
\footnote{$j(\nu)$, $j(t)$ and $\b(\nu)$, $\b(t)$ are different functions and not the same function with different arguments.}
\beq
j(\nu)\, \to\,  j(t)\,, \qquad \lambda^{\frac{j(\nu)-1}{2}}\b(\nu)\, \to\,  \frac{1}{2 \pi} \,\b(t)\,.
\label{eq:lim_beta}
\eeq
The power of $\lambda$ in the relation of $\beta$'s is necessary to cancel the powers of $\lambda$ in the relation between $S$ and $\alpha' s$.
It is compatible with the expectation that each derivative in the couplings of the spin $J$ operators forming the Pomeron comes at least with a power of $\lambda^{-\frac{1}{4}}$.

Next we consider the optical theorem in AdS \eqref{eq:gluing_stripped} and flat space \eqref{eq:impact_optical_theorem}
\beq
- \Re \cB_{\text{1-loop}} = \frac{1}{2}
\sum\limits_{\substack{\De_5,\rho_5\\\De_6,\rho_6}}
\cB^{3652\, *}_\text{tree}  \cB^{1564}_\text{tree}
\ \to \ 
		\Im \de_{\text{1-loop}} (b) 
		= \frac{1}{2} \sum\limits_{\substack{m_5,\rho_5,\e_5\\m_6,\rho_6,\e_6}}
		\de_\text{tree}^{3652\,*} \de_\text{tree}^{1564}\,.
\eeq
The similarity is striking, however we have to make sure  the sums and summands are in fact related by the flat space limit.
The additional sums over polarizations can be evaluated using completeness relations such as 
\eqref{eq:completeness_relation}, which evaluate to contractions just as in the AdS equation.
We also have to make sure that the labels $\rho$ on both sides are irreducible representations of the same group $SO(d)$. This is indeed the case for massive particles if we consider the flat space limit $AdS_{d+1} \to \mathbb{R}^{1,d}$, which has the massive Little group $SO(d)$.

The next step is to match the tree-level correlators \eqref{eq:Btree_differential} and amplitudes \eqref{eq:A_tree_regge} that involve spinning particles 5 and 6. In this case the flat space limit gives
\begin{align}
&\cB^{(\De_5,\rho_5),(\De_6,\rho_6)}_{\mathbf{m} \mathbf{n}}  (p,\bar{p}) 
		= 4 \pi i \int\limits_{0}^\infty d\nu \, S^{j(\nu)-1} 
		\,\mathfrak{D}^{(\De_5,\rho_5),(\De_6,\rho_6)}_{\mathbf{m} \mathbf{n}} (\nu)
		\,\Omega_{i \nu} (L)\ \to
		\label{eq:flat_space_limit_spinning}
\\
\to\ &
i \de^{(m_5,\rho_5),(m_6,\rho_6)}_{\mathbf{m} \mathbf{n}} (s,b) 
= i \int\limits_{\mathbb{R}^{D-2}} \frac{dq}{(2\pi)^{D-2}} \left(\frac{\alpha's}{4}\right)^{j(t)-1} A^{12P}_{m_5,\rho_5,\mathbf{m}} (q,v)\, \beta(t) A^{34P}_{m_6,\rho_6,\mathbf{n}}(q,v)\, e^{i q\cdot b}
\nonumber \\
& \qquad \qquad = 2i \int\limits_0^{\oo} d|q| \left(\frac{\alpha's}{4}\right)^{j(t)-1} A^{12P}_{m_5,\rho_5,\mathbf{m}} (-i\partial_{b},v)\, \beta(t) A^{34P}_{m_6,\rho_6,\mathbf{n}}(-i\partial_{b},v)\, \omega_q(b)\,,
\nonumber
\end{align}
where the derivative $\partial_{b}$ is with respect to the components of the transverse vector $b$.
The difference compared to \eqref{eq:flat_space_limit_dilatons} is that in AdS we  have differential operators that generate tensor structures, while  in flat space the tensor structures are the ones of the on-shell three-point amplitudes. As discussed in section \ref{sec:3pt_amplitudes}, these three-point amplitudes are given in terms of the Pomeron momentum $q$ and, for massive particles, the longitudinal polarization vector $v$, which is transverse to $q$.
We will now study the relation of these two kinds of tensor structures to the flat space limit.

We begin with the covariant derivatives and will argue that they become derivatives in  impact parameter in flat space, i.e.
	\beq
		\nabla_p^{m} \Omega_{i \nu} (L)  \ \to\  
		R \partial_b^m e^{i b \cdot q} = R i q^m e^{i b \cdot q}\,.
		\label{eq:nabla_fsl}
	\eeq
In order to show this, we will act with two contracted covariant derivatives either on a single harmonic function or on two different ones, covering all situations that can occur.
Acting on a single harmonic function we obtain, from \eqref{eq:nabla} and \eqref{eq:flat_space_limit},
	\beq
		\frac{1}{\sqrt{\l}}\, \nabla_p^2 \, \Omega_{i \nu_{1,2}} (L) \ \to\  - \alpha' q_{1,2}^2\, \omega_{i \nu_{1,2}} (b)\,.
\label{eq:laplace_flat_space_limit}
	\eeq
The action of contracted covariant derivatives on two different harmonic functions is captured by the functions $W_k$ in \eqref{eq:Wk}, which is given in the flat space limit by the leading term \eqref{eq:Wk_leading}
	\bea
		\frac{W_{k} \big(\nu_1^2,\nu_2^2,\nu^2\big)}{\big(\sqrt{\l}\big)^k}\  \to\ 
		\left( \frac{\nu_1^2+\nu_2^2 - \nu^2}{2\sqrt{\l}} \right)^k
		\ \to \ \left(  \alpha'\, \frac{q_1^2+q_2^2 - q^2}{2} \right)^k
		= (\alpha')^{k} (- q_{1}\cdot q_2)^k\,, 
	\eea{eq:Wk_fs}
where we used that $q = q_1 + q_2$. This implies that the flat space limit of \eqref{eq:Wk} is
	\begin{align}
	&
\frac{1}{\big(\sqrt{\l}\big)^k}
		{\nabla_{p}}_{m_1} \ldots {\nabla_{p}}_{m_k}   \Omega_{i \nu_1} (L)
		\nabla_p^{m_1} \ldots \nabla_p^{m_k} \Omega_{i \nu_2} (L)
		\ \to
		\\
		\to \ & 
		(-\a')^k q_{1 \, m_1} \ldots q_{1 \, m_k}   e^{i b \cdot q_1}
		q_2^{m_1} \ldots q_2^{m_k} e^{i b \cdot q_2}\,.
\label{eq:cov_flat_12}
	\end{align}
We conclude that both \eqref{eq:laplace_flat_space_limit} and \eqref{eq:cov_flat_12} are compatible with \eqref{eq:nabla_fsl}.
Apart from the covariant derivative, tensor structures depend also on the direction $\hat{p}$, which is normal to the transverse space $H_{d-1}$ and satisfies $\hat{p}^2=-1$. In flat space the only possible direction for polarizations that is normal to the transverse space is the unit vector $v$, hence we have to require that in the flat space limit
	\beq
		\hat{p}^m \,\to\, i v^{m}\,.
		\label{eq:phat_fsl}
	\eeq

With the identifications \eqref{eq:nabla_fsl} and \eqref{eq:phat_fsl}, the matching in \eqref{eq:flat_space_limit_spinning} works provided that the spectral functions $\beta^{k_5,k_6}_{(\De_5,\rho_5),(\De_6,\rho_6)} (\nu)$ in \eqref{eq:Dfrak} are such that
	\beq
		\lambda^{\frac{j(\nu)-1}{2}}\mathfrak{D}^{(\De_5,\rho_5),(\De_6,\rho_6)}_{\mathbf{m} \mathbf{n}} (\nu)\, \Omega_{i \nu} (L) \, \to\, 
\frac{1}{2 \pi}
		A^{12P}_{m_5,\rho_5,\mathbf{m}} (-i\partial_{b},v) \,\beta(t)\, A^{34P}_{m_6,\rho_6,\mathbf{n}} (-i\partial_{b},v)
		\,\omega_{q}(b)\,.
\label{eq:flat_space_limit_ts}
	\eeq
Using the explicit tensor structures for three-point amplitudes in \eqref{eq:3pt_spinning}, the matching \eqref{eq:flat_space_limit_ts} can also be expressed for the spectral function for any given tensor structure
\beq
 \lambda^{\frac{j(\nu)-1}{2}} \lambda^{\frac{|\rho_5|-k_5}{4}}\lambda^{\frac{|\rho_6|-k_6}{4}}\beta^{k_5,k_6}_{(\De_5,\rho_5),(\De_6,\rho_6)} (\nu)
\ \to\ 
\frac{1}{2 \pi}\,
a_{m_5,\rho_5}^{k_5}(t) \,a_{m_6,\rho_6}^{k_6}(t) \,\b(t)\,.
\label{eq:flat_space_limit_beta_spinning}
\eeq
The powers of $\lambda$ are again compatible with a factor of $\lambda^{-\frac{1}{4}}$ in $\beta^{k_5,k_6}_{(\De_5,\rho_5),(\De_6,\rho_6)} (\nu)$ for each derivative in the coupling.
Such a scaling is expected from the general arguments of \cite{Costa:2017twz,Meltzer:2017rtf}.
We have now shown that all tree-level phase shifts appearing in the AdS and flat space optical theorems can be related by the flat space limit. 

Let us also compare the vertex functions that appear in both  AdS and flat space impact parameter representations of the one-loop amplitudes. 
The flat space limit of   \eqref{eq:Bt_SL_vertex}  is given by the impact parameter transform of \eqref{eq:optical_theorem_V_flat}, i.e.
\begin{equation}
- \Re \cB_{\text{1-loop}} (p,\pb) \ \to\  \Im \delta_{\text{1-loop}} (s, b)\,,
\end{equation}
becomes
\bea
		& 2 \pi^2  \int\limits_{-\infty}^\infty d\nu d\nu_1 d\nu_2 \, \beta(\nu_1)^* \beta(\nu_2)
		  \, V(\nu_1,\nu_2,\nu)^2 S^{j(\nu_1)+j(\nu_2)-2} \Phi(\nu_1,\nu_2,\nu) \, \Omega_{i \nu} (L) \ \to\\
\to \ & \frac{1}{2}
\int\limits_{\mathbb{R}^{D-2}} \frac{d q d q_1  d q_2}{(2\pi)^{2(D-2)}} 
\, \b(t_1)^* \b(t_2) \, V(q_1,q_2)^2 \left(\frac{\alpha's}{4}\right)^{j(t_1) + j(t_2)-2} 
\de(q-q_1-q_2) \, e^{i q\cdot b}\,.
\eea{eq:flat_space_limit_with_V}
In this case we can use the delta function to write all the other scalar functions in terms of $q_1^2$, $q_2^2$ and $q^2$, however we need to do the angular integral over the delta function itself. To this end we can define
\beq
\int\limits_{\mathbb{R}^{D-2}}  \frac{dq_1 dq_2}{(2\pi)^{D-2}} \, \de(q-q_1-q_2)
= 4\int\limits_{0}^{\oo} d|q_1| d|q_2|\, \phi(q_1, q_2, q)\,.
\label{eq:flat_phi}
\eeq
Using this and \eqref{eq:exp_to_harmonic}, we can compute the angular integrals in
	\beq
\int\limits_{\mathbb{R}^{D-2}}  \frac{dq_1 dq_2}{(2\pi)^{2(D-2)}} \, e^{i b \cdot (q_1 + q_2)} = \int\limits_{\mathbb{R}^{D-2}}  \frac{dq_1 dq_2 dq}{(2\pi)^{2(D-2)}} \, \de(q-q_1 -q_2)\, e^{i b \cdot q} \,,		\label{eq:Wk_flat}
	\eeq
to find the flat space version of \eqref{eq:Omega_prod}
\beq
\omega_{q_1}(b)\,  \omega_{q_2}(b) = 2 \int\limits_{0}^{\infty} d|q| \, \phi(q_1, q_2, q) \, \omega_{q}(b)\,.
\eeq
Using the explicit expressions for $\Phi$ and $\phi$ (which can be found for instance in appendix E of  \cite{Penedones:2010ue}), one can further check that under the flat space limit
\beq
R^{4-D} \Phi(\nu_1, \nu_2, \nu) \ \to\  \phi(q_1, q_2, q)\,.
\eeq
With this relation is clear that the expressions in
\eqref{eq:flat_space_limit_with_V} are indeed related by the flat space limit provided that the vertex functions are related by
\beq
V(\nu_1,\nu_2,\nu)\  \to\  V(q_1,q_2) = V(t_1, t_2, t)\,.
\eeq

%%%%%%%%%%%%%%%%%%%%%%%%%%%%%%%%%%%%%%%%%%%%%%%%%%%%%%%%%%%%%%%%%%%%%%%%%%%%%%%%%
\subsection{Constraining AdS quantities}
\label{eq:constraining_from_flat_space_limit}
%%%%%%%%%%%%%%%%%%%%%%%%%%%%%%%%%%%%%%%%%%%%%%%%%%%%%%%%%%%%%%%%%%%%%%%%%%%%%%%%%

We saw above that all elements of the impact parameter optical theorems in AdS and flat space are related by the flat space limit provided that $j(\nu)$, $\beta(\nu)$, $\beta^{k_5,k_6}_{(\De_5,\rho_5),(\De_6,\rho_6)} (\nu)$ and $V\big(\nu_1^2,\nu_2^2,\nu^2\big)$ are given by their flat space counterparts in the limit.
In this subsection, we briefly review how the limit actually constrains these functions.  All of these objects depend on two dimensionless quantities, the spectral parameters $\nu$ and the t'Hooft coupling $\lambda$.  Let us discuss this for a generic function $f(\nu)$ that is required to satisfy the flat space limit
\beq
f(\nu,\lambda) \, \to \,  f(t)\,.
\eeq
Existence of the  gravity limit requires the function to have an expansion in negative powers of $\sqrt{\lambda}$ of the form
\beq
f(\nu,\lambda) = \sum\limits_{n=0}^{\infty} \frac{f_n(\nu)}{\lambda^{n/2}}\,.
\eeq
In order for the flat space limit \eqref{eq:flat_space_limit} of the function to be finite, the functions $f_n(\nu)$ must  have an expansion in large $\nu$ with leading power not larger than $2n$,
\beq
f_n(\nu) = a_{n,n} \nu^{2n} + a_{n,n-1} \nu^{2(n-1)} + a_{n,n-2} \nu^{2(n-2)} + \ldots\,,
\eeq
which ensures finiteness of the limit order by order in the large $\lambda$ expansion.
The flat space limit of $f(\nu,\lambda)$ is then
\beq
f(\nu,\lambda)\  \to\  \sum\limits_{n=0}^{\infty} a_{n,n} \left(\frac{\nu^{2}}{\sqrt{\lambda}}\right)^n \to\
\sum\limits_{n=0}^{\infty} a_{n,n} (\a' q^2)^n\,.
\eeq
At every order in $1/\sqrt{\l}$ the leading power of $\nu$ survives and is fixed by the flat space limit, while all the other powers are subleading, and cannot be determined from this condition. 
These considerations hold for $j(\nu)$, $\beta(\nu)$, $\beta^{k_5,k_6}_{(\De_5,\rho_5),(\De_6,\rho_6)} (\nu)$ and $V(\nu_1,\nu_2,\nu)$, fixing part of these functions. These facts have been explored in detail for the functions  $j(\nu)$ and $\beta(\nu)$ in \cite{Cornalba:2008qf,Costa:2012cb}.



