%!TEX root = ../Central_compile.tex
%%%%%%%%%%%%%%%%%%%%%%%%%%%%%%%%%%%%%%%%%%%%%


%%%%%%%%%%%%%%%%%%%%%%%%%%%%%%%%%%%%%%%%%%%%%%%%%%%%%%%%%%%%%%%%%%%%%%%%%%%%%%%%%%%%%%%%%%
\section{Branching relations for projectors}
\label{sec:branchingprojs}
%%%%%%%%%%%%%%%%%%%%%%%%%%%%%%%%%%%%%%%%%%%%%%%%%%%%%%%%%%%%%%%%%%%%%%%%%%%%%%%%%%%%%%%%%%

In this Appendix we provide a detailed account of all the branching relations for closed string state projectors utilized in section \ref{sec:massive_tree_flat}.
Let us start by recalling the SO(9) closed string states at level 1
\begin{align}
&\left(\ydiagram{2}_{\,9} \oplus \, \ydiagram{1,1,1}_{\,9}\right)^{2} = 
      \ydiagram{2,2,2}_{\,9}
\oplus \,      \ydiagram{2,2,1,1}_{\,9}
\oplus \, 2 \, \ydiagram{3,1,1}_{\,9}
\oplus \, 3 \, \ydiagram{2,1,1,1}_{\,9}
\oplus \,      \ydiagram{4}_{\,9}
\oplus \,      \ydiagram{3,1}_{\,9}
\nonumber\\
&\oplus \, 2 \, \ydiagram{2,2}_{\,9}
\oplus \,      \ydiagram{2,1,1}_{\,9}
\oplus \,      \ydiagram{1,1,1,1}_{\,9}
\oplus \, 2 \, \ydiagram{2,1}_{\,9}
\oplus \, 3 \, \ydiagram{1,1,1}_{\,9}
\oplus \, 2 \, \ydiagram{2}_{\,9}
\oplus \, 2 \, \ydiagram{1,1}_{\,9}
\oplus \, 2 \, \bullet
\,.
\label{eq:closed_states_level_1}
\end{align}
We recall that states with more than 3 columns will not contribute as we only have 3 different vectors to anti-symmetrize. We are going to perform the branching
\beq
\text{SO(9)} ~\rightarrow~ \text{SO(4)}\times \text{SO(5)}|_\bullet \,,
\eeq
where we denote the projection to singlets of SO(5) by $|_\bullet$. Note that certain representations can naturally be restricted to SO(4) by simply taking the 5d subset of 10d indices. It is obvious that
\beq
\ydiagram{1,1}_{\,9}\,\,\rightarrow\,\, \ydiagram{1,1}_{\,4}\,,\qquad \bullet_{\,9} \,\, \rightarrow \,\, \bullet_{\,4}\,.
\eeq
Additionally, the projector for these representations is identical for SO(9) and SO(4) (up to restriction of the indices $\alpha \to a \,,\, \beta \to b \,,\, \mu \to m \,,\, \dots$)
\beq
\pi_{[1,1]_9}^{a_1a_2;b_1b_2} = \pi_{[1,1]_4}^{a_1a_2;b_1b_2}= \frac{1}{2}\left(\eta^{a_1 b_1}\eta^{a_2 b_2}-\eta^{a_1 b_2}\eta^{a_2 b_1} \right) \,. 
\eeq
Other representations admit a direct restriction, but can also give additional irreps, by the creation of SO(5) singlets, through the contraction of indices with legs on the compact manifold. For example the spin 2 states branch as
\beq
\ydiagram{2}_{\,9}\,\,\rightarrow\,\, \ydiagram{2}_{\,4} \,\, \oplus \,\, \bullet_{\,4}\,.
\eeq
The spin 2 on the RHS is interpreted as the restriction of indices to the SO(4) and the singlet as a trace over the compact space indices. In terms of projectors the statement is simply
\beq
\pi_{[2]_9}^{a_1a_2;b_1b_2} = \pi_{[2]_4}^{a_1a_2;b_1b_2}+ \frac{5}{36}\eta^{a_1 a_2}\eta^{b_1 b_2}\,.
\eeq
Similarly, for the spin 4 case
\beq
\ydiagram{4}_{\,9}\,\,\rightarrow\,\, \ydiagram{4}_{\,4} \,\, \oplus \,\, \ydiagram{2}_{\,4} \,\, \oplus \,\,\bullet_{\,4},
\eeq
and the projector equation is
\beq
\pi_{[4]_9}^{\mathbf{a};\mathbf{b}}
= \sum_{\rho =[4]_4,[2]_4,\bullet_4}
b^{\mathbf{a}}_{[4]_9 \to \rho,\mathbf{m}} \ 
\pi_{\rho}^{\mathbf{m};\mathbf{n}} \ 
b^{\mathbf{b}}_{[4]_9 \to \rho,\mathbf{n}}\,.
\eeq
It will be again convenient to manifestly symmetrize the tensors, in order to present them more compactly. We define
\beq
b^{\mathbf{a}}_{\rho_C \to \rho,\mathbf{m}}\equiv \pi^{\mathbf{a}}_{\rho_C, \mathbf{a}'} \tilde{b}^{\mathbf{a'}}_{\rho_C \to \rho,\mathbf{m}}\,,
\label{eq:branching_tensors_symmetrization}
\eeq
and then present a list of the simpler $\tilde{b}$. In this case we have
\bea
\tilde{b}^{a_1 a_2 a_3 a_4}_{[4]_9 \to [4]_4,m_1 m_2 m_3 m_4}&= \delta^{a_1}_{m_1}\delta^{a_2}_{m_2}\delta^{a_3}_{m_3}\delta^{a_4}_{m_4}\equiv \delta^ {\mathbf{a}}_{\mathbf{m}}\\
\tilde{b}^{a_1 a_2 a_3 a_4}_{[4]_9 \to [2]_4,m_1 m_2 }&= \sqrt{\frac{39}{20}}\delta^{a_1}_ {m_1}\delta^{a_2}_ {m_2}\eta^{a_3 a_4} ~,~ \tilde{b}^{a_1 a_2 a_3 a_4}_{[4]_9 \to \bullet_4}
  = \sqrt{\frac{143}{280}}\eta^{a_1a_2}\eta^{a_3 a_4}\,.
\eea{eq:branch4}
The fact that the direct restriction of the irrep $[4]_9 \rightarrow[4]_4$ comes with coefficient 1 is a non-trivial consistency check of the previous procedure.

There are other irreps that don't admit a direct restriction, because they have more than two columns (SO(4) Young tableaux have at most two columns, and traces can vanish by antisymmetry). For this we need to use the SO(4) Levi-Civita tensor. We will simply write it as $\varepsilon^{a_1a_2a_3a_4}$. The simplest case is the 3-form
\beq
\ydiagram{1,1,1}_{\,9}\,\,\rightarrow\,\, \ydiagram{1}_{\,4}\,,
\eeq
and the corresponding projector equation is
\beq
\pi_{[1,1,1]_9}^{\mathbf{a};\mathbf{b}} =\frac{1}{6} \varepsilon^{a_1 a_2 a_3}_{\,\,\,\,\qquad m} \pi_{[1]_4}^{m;n} \, \varepsilon^{\,\,\,b_1 b_2 b_3}_{n}= 4\, \pi_{[1,1,1,1]_4}^{\mathbf
{a} m;\mathbf{b} n}\,\pi_{[1]_4,m;n}\,,
\label{eq:111_to_1}
\eeq
From the first equation we can read off
\beq
b^{a_1 a_2 a_3}_{[1,1,1]_9\to[1]_4,m}= \sqrt{\frac{1}{6}}\varepsilon^{a_1 a_2 a_3}_{\,\,\,\,\qquad m}\,.
\eeq
For the second equality in (\ref{eq:111_to_1}) we used the standard identity
\beq
\epsilon^{a_1a_2a_3a_4}\epsilon^{b_1b_2b_3b_4}= 4!\, \pi_{[1,1,1,1]_4}^{\mathbf{a};\mathbf{b}}\,.
\eeq
This is convenient to square the three-point amplitudes when computing the residue of the four-point function $A^{12P_1P_2}$.
Using trace subtractions and products of epsilon tensors we can now derive branching identities for all the relevant irreps. Let us list the remaining identities, where we write the trace subtractions and the Levi-Civita tensors using the trial projectors $\tilde{b}$, but then appropriately symmetrize them through (\ref{eq:branching_tensors_symmetrization})
\bea
\ydiagram{2,1,1}_{\,9}\,\,&\rightarrow\,\, \ydiagram{2}_{\,4} \,\, \oplus \,\,  \ydiagram{1,1}_{\,4}\,,\\
\tilde{b}^{a_1 a_2 a_3 a_4}_{[2,1,1]_9\to[2]_4,m_1 m_2}&=\sqrt{\frac{6}{4!}}  \varepsilon^{a_1a_3a_4}_{\qquad m_1}\delta^{a_2}_{m_2} ~,~ \tilde{b}^{a_1 a_2 a_3 a_4}_{[2,1,1]_9\to[1,1]_4,m_1 m_2}=\sqrt{\frac{42}{4!5}} \varepsilon^{a_1a_3a_4}_{\,\,\,\,\qquad m_1} \delta^{a_2}_{m_2}\,,\\
\ydiagram{2,1}_{\,9}\,\,&\rightarrow\,\, \ydiagram{2,1}_{\,4} \,\, \oplus \,\,  \ydiagram{1}_{\,4}\,,\\
\tilde{b}^{\mathbf{a
}}_{[2,1]_9\to[2,1]_4,\mathbf{m}}&= \delta^{\mathbf{a}}_{\mathbf{m}} ~,~ \tilde{b}^{a_1 a_2 a_3}_{[2,1]_9\to[1]_4,m} = \sqrt{\frac{16}{5}} \delta^{a_1}_{m_1} \eta^{a_2 a_3}\,,\\
\eea{eq:branching1}

\bea
\ydiagram{3,1,1}_{\,9}\,\,&\rightarrow\,\, \ydiagram{3}_{\,4} \,\, \oplus \,\,  \ydiagram{2,1}_{\,4}\,\, \oplus \,\,  \ydiagram{1}_{\,4}\,,\\
\tilde{b}^{\mathbf{a}}_{[3,1,1]_9\to[3]_4,\mathbf{m}} &= \sqrt{\frac{36}{4!5}} \varepsilon^{a_1a_4a_5}_{\,\,\,\, \qquad m_1} \delta^ {a_2}_{m_2} \delta^ {a_3}_{m_3} ~,~ \tilde{b}^{\mathbf{a}}_{[3,1,1]_9\to[2,1]_4,\mathbf{m}} = \sqrt{\frac{1152}{4!25}} \varepsilon^{a_1a_4a_5}_{\,\,\,\, \qquad m_1} \delta^ {a_2}_{m_2} \delta^ {a_3}_{m_3}\,,\\
\tilde{b}^{a_1 a_2 a_3 a_4 a_5}_{[3,1,1]_9\to[1]_4,m} &= \sqrt{\frac{22}{4! 5}} \varepsilon^{a_1a_4a_5}_{\,\,\,\, \qquad m} \eta^{a_2 a_3}\,,\\
\ydiagram{3,1}_{\,9}\,\,&\rightarrow\,\, \ydiagram{3,1}_{\,4} \,\, \oplus \,\,  \ydiagram{1,1}_{\,4}\,\, \oplus \,\,  \ydiagram{2}_{\,4}\,,\\
\tilde{b}^{\mathbf{a
	}}_{[3,1]_9\to[3,1]_4,\mathbf{m}}&= \delta^{\mathbf{a}}_{\mathbf{m}}, \tilde{b}^{\mathbf{a
}}_{[3,1]_9\to[1,1]_4,\mathbf{m}}= \sqrt{\frac{11}{10}} \delta^{a_1}_{m_1} \eta^{a_2 a_3} \delta^{a_4}_{m_2}, \tilde{b}^{\mathbf{a
}}_{[3,1]_9\to[2]_4,\mathbf{m}}= \sqrt{\frac{27}{10}} \delta^{a_1}_{m_1} \eta^{a_2 a_4} \delta^{a_3}_{m_2},\\
\ydiagram{2,2}_{\,9}\,\,&\rightarrow\,\, \ydiagram{2,2}_{\,4} \,\, \oplus \,\,  \ydiagram{2}_{\,4}\,\, \oplus \,\, \bullet_{\,4}\,,\\
\tilde{b}^{\mathbf{a
	}}_{[2,2]_9\to[2,2]_4,\mathbf{m}}&= \delta^{\mathbf{a}}_{\mathbf{m}} ~,~
\tilde{b}^{\mathbf{a
	}}_{[2,2]_9\to[2]_4,\mathbf{m}}= \sqrt{\frac{42}{5}}  \delta^{a_1}_{m_1} \eta^{a_2 a_3} \delta^{a_4}_{m_2} ~,~\tilde{b}^{\mathbf{a
}}_{[2,2]_9\to \bullet_4}= \sqrt{\frac{7}{5}}\eta^{a_1 a_4}\eta^{a_2 a_3} \,,\\
\ydiagram{2,2,2}_{\,9}\,\,&\rightarrow\,\, \ydiagram{2}_{\,4} \,\, \oplus \,\, \bullet_{\,4}\,,\\
\tilde{b}^{\mathbf{a
	}}_{[2,2,2]_9\to[2]_4,m_1 m_2}&= \frac{\sqrt{48}}{4!} \varepsilon^ {a_1 a_5 a_3}_{\,\,\,\,\qquad m_1} \varepsilon^ {a_4 a_2 a_6}_{\,\,\,\,\qquad m_2} ~,~
\tilde{b}^{\mathbf{a
		}}_{[2,2,2]_9\to\bullet_4}= \frac{\sqrt{28}}{4!} \varepsilon^ {a_1 a_5 a_3}_{\,\,\,\,\qquad m} \varepsilon^ {a_4 a_2 a_6 m}  \,.
\eea{eq:listofbranchings}
Note that for the last diagram, which has more than 2 boxes in both columns, we are forced to use two pairs of epsilon tensors. Tensors with more than three rows aren't allowed by the 10d kinematics, but even if they were, their branchings do not contain singlets of SO(5) so we can simply discard them.
\subsection{All 5d closed string amplitudes}
\label{sec:all_5d_closed}
 Let us first write down in generality the 5d amplitudes using the relations derived in the main text. We have
\beq
A^{15P_1}_{n,\rho_L \otimes \rho_R \to \rho_C \to \rho,\mathbf{m}} = 
 A^{15P_1}_{\r_L,\balpha} \ A^{15P_1}_{\r_R,\bbeta} \
p^{\balpha \bbeta}_{\rho_L \otimes \rho_R \to \rho_C,\mathbf{a}}
b^{\mathbf{a}}_{\rho_C \to \rho,\mathbf{m}}\,.
\eeq
With all the group theory identities established, we can enumerate all the amplitudes used in the main text to reproduce the residue at the cut with mass level 1. However, we again emphasize that we have not explicitly symmetrized the amplitudes by contracting with the respective projector, in order to maintain some compactness of the tables below. Namely, all amplitudes are to be contracted with the projector to the SO(4) irrep, and furthermore, amplitudes where $\text{rank}(\rho)=\text{rank}(\rho_L)+\text{rank}(\rho_R)$ are also not explicitly symmetrized with respect to $\rho_L$ and $\rho_R$ as in equation (\ref{eq:symmetrize_factors}). 
Additionally, for amplitudes where the $b$ tensors contain a Levi-Civita symbol, we write the square of the amplitude
\beq
\left( A_{\mathbf{m}}^{\rho_L\otimes \rho_R \to \rho_C \rightarrow \rho}\right) ^2 \equiv
A_{\mathbf{m}}^{\rho_L\otimes \rho_R \to \rho_C \rightarrow \rho}\pi^ {\mathbf{m};\mathbf{n}}_{\rho}A_{\mathbf{n}}^{\rho_L\otimes \rho_R \to \rho_C \rightarrow \rho}\,,
\eeq
since the amplitude itself cannot be written nicely in terms of $v_\alpha, q_\alpha,\epsilon_\alpha$. 
With these caveats in mind, we list the amplitudes starting by the ones with the most indices\\
\bea
\ydiagram{4}_{\,4}\\
A^{[2]_9\otimes[2]_9 \rightarrow[4]_9 \rightarrow[4]_4}_{m_1 m_2 m_3 m_4}&=
\frac{1}{8} \alpha ' \left(2 \epsilon _{1,m _1} q_{1,m _2}+v_{m _1} v_{m _2} q_1\cdot \epsilon _1\right)\\ &\times
\left(2 \epsilon _{1,m _3} q_{1,m _4}+v_{m _3} v_{m _4} q_1\cdot \epsilon _1\right) \,,\\
&&\\
2 \,\ydiagram{2,2}_{\,4}\\
A^{[2]_9\otimes[2]_9 \rightarrow[2,2]_9 \rightarrow[2,2]_4}_{m_1 m_2 m_3 m_4}&=
\frac{1}{8} \alpha ' \left(2 \epsilon _{1,m _1} q_{1,m _2}+v_{m _1} v_{m _2} q_1\cdot \epsilon _1\right)\\ &\times
\left(2 \epsilon _{1,m _3} q_{1,m _4}+v_{m _3} v_{m _4} q_1\cdot \epsilon _1\right)\,,\\
&&\\
A^{[1,1,1]_9\otimes[1,1,1]_9 \rightarrow[2,2]_9 \rightarrow[2,2]_4}_{m_1 m_2 m_3 m_4}&=\frac{\alpha'}{4 \sqrt{15}} [ \left(v_{m _1} \epsilon _{1,m _3}-v_{m _3} \epsilon _{1,m _1}\right) \left(q_1\cdot
\epsilon _1 \left(v_{m _2} q_{1,m _4}-v_{m _4} q_{1,m _2}\right)\right. \\
&+t_1 \left.\left(v_{m _2} \epsilon
_{1,m _4}-v_{m _4} \epsilon _{1,m _2}\right)\right)\\&+q_{1,m _1} \left(\epsilon _{1,m _3} \left(\epsilon
_{1,m _2} q_{1,m _4}-\epsilon _{1,m _4} q_{1,m _2}\right)\right.\\
&+\left.v_{m _3} \left(v_{m _4} \left(\epsilon
_{1,m _2} q_1\cdot \epsilon _1-q_{1,m _2}\right)+v_{m _2} \left(q_{1,m _4}-\epsilon _{1,m _4} q_1\cdot
\epsilon _1\right)\right)\right)\\
&+q_{1,m _3} \left(\epsilon _{1,m _1} \left(\epsilon _{1,m _4} q_{1,m
	_2}-\epsilon _{1,m _2} q_{1,m _4}\right)\right.\\
&+\left.v_{m _1} \left(v_{m _4} \left(q_{1,m _2}-\epsilon _{1,m
	_2} q_1\cdot \epsilon _1\right)+v_{m _2} \left(\epsilon _{1,m _4} q_1\cdot \epsilon _1-q_{1,m
_4}\right)\right)\right)]\,,\\
&\\
\ydiagram{3}_{\,4}\\
\left( A_{\mathbf{m}}^{[2]_9\otimes[111]_9 \rightarrow[311]_9 \rightarrow[3]_4}\right)^2 &=-\frac{\left(\alpha '\right)^2}{1152}\left(48 \left(q_1\cdot \epsilon _2\right){}^2 \left(3
	\left(q_2\cdot \epsilon _1\right){}^2+t_2\right)\right.\\&+48 t_1 \left(\left(q_2\cdot \epsilon
	_1\right){}^2+t_2 \left(\epsilon _1\cdot \epsilon _2\right){}^2\right)+48 \left(q_1\cdot
	q_2\right){}^2 \left(1-3 \left(\epsilon _1\cdot \epsilon _2\right){}^2\right)\\&+115
	(q_1\cdot q_2) (\epsilon _1\cdot \epsilon _2) (q_1\cdot \epsilon _1) (q_2\cdot \epsilon _2)\\ &\left.-115
	(q_1\cdot \epsilon _1) (q_1\cdot \epsilon _2) (q_2\cdot \epsilon _1) (q_2\cdot \epsilon
	_2)\right)\,,\\
&\\
2\,\ydiagram{1}_{\,4}\\
\left( A_{\mathbf{m}}^{[2]_9\otimes[111]_9 \rightarrow[311]_9 \rightarrow[1]_4}\right)^2&=-\frac{625 \left(\alpha '\right)^2}{12672} (q_1\cdot \epsilon _1) (q_2\cdot \epsilon _2)\\&\times \left((q_1\cdot
	\epsilon _2) (q_2\cdot \epsilon _1)-(q_1\cdot q_2) (\epsilon _1\cdot \epsilon
	_2)\right)\,,\\
&\\
\left( A_{\mathbf{m}}^{[2]_9\otimes[111]_9 \rightarrow[111]_9 \rightarrow[1]_4}\right)^2&=\frac{65 \left(\alpha '\right)^2}{1408} (q_1\cdot \epsilon _1) (q_2\cdot \epsilon _2)\\&\times \left((q_1\cdot
	q_2) (\epsilon _1\cdot \epsilon _2)-(q_1\cdot \epsilon _2) (q_2\cdot \epsilon _1)\right)\,.
\eea{eq:list_amplitudes_flat}
And
\bea
2\,\ydiagram{2,1}_{\,4}\\
A_{m_1 m_2 m_3}^{[2]_9\otimes[111]_9 \rightarrow[21]_9 \rightarrow[21]_4}&=
\frac{\alpha'}{12
	\sqrt{6}}[ \left(q_{1,m _1} \left(v_{m _3} \left(2 \epsilon _{1,m _2} q_1\cdot \epsilon _1-3 q_{1,m
	_2}\right)\right.\right.\\&+\left.v_{m _2} \left(3 q_{1,m _3}-2 \epsilon _{1,m _3} q_1\cdot \epsilon _1\right)\right)\\&+2 q_1\cdot
\epsilon _1 \left(q_{1,m _3} \left(v_{m _2} \epsilon _{1,m _1}-v_{m _1} \epsilon _{1,m
	_2}\right)\right.\\&+\left.q_{1,m _2} \left(v_{m _1} \epsilon _{1,m _3}-v_{m _3} \epsilon _{1,m _1}\right)\right)\\&+\left.3
t_1 \epsilon _{1,m _1} \left(v_{m _2} \epsilon _{1,m _3}-v_{m _3} \epsilon _{1,\alpha _2}\right)\right)]\,,\\
&\\
\left( A_{\mathbf{m}}^{[2]_9\otimes[111]_9 \rightarrow[311]_9 \rightarrow[21]_4}\right)^2&=\frac{5 \left(\alpha '\right)^2 }{1152}\left(6 t_2 \left(q_1\cdot \epsilon _2\right){}^2+6 t_1
\left(\left(q_2\cdot \epsilon _1\right){}^2+t_2 \left(\epsilon _1\cdot \epsilon
_2\right){}^2\right)\right.\\&+(q_1\cdot q_2) (\epsilon _1\cdot \epsilon _2) (q_1\cdot \epsilon _1)
(q_2\cdot \epsilon _2)\\&-\left.(q_1\cdot \epsilon _1) (q_1\cdot \epsilon _2) (q_2\cdot \epsilon _1)
(q_2\cdot \epsilon _2)+6 \left(q_1\cdot q_2\right){}^2\right)\,,\\
&\\
\eea{eq:21diagrams}
We also have the sixfold degenerate spin 2 states
\bea
6\,\ydiagram{2}_{\,4}\\
A_{m_1 m_2}^{[2]_9\otimes[2]_9\rightarrow[4]_9\rightarrow[2]_4}&= \frac{\alpha '}{16} \sqrt{\frac{5}{39}}  \left[q_{1,m _1} \left(5 \epsilon _{1,m _2} (q_1\cdot \epsilon _1)+2 q_{1,m
	_2}\right)\right.\\ &+\epsilon _{1,m _1} \left(5 q_{1,m _2} (q_1\cdot \epsilon _1)-2 t_1 \epsilon _{1,m _2}\right)\\&\left.+5
v_{m _1} v_{m _2} \left(q_1\cdot \epsilon _1\right){}^2\right]\,,\\
&\\
A_{m_1 m_2}^{[2]_9\otimes[2]_9\rightarrow[2,2]_9\rightarrow[2]_4}&= \frac{\alpha '}{4} \sqrt{\frac{5}{42}}  \left[-q_{1,m _1} \left(q_{1,m _2}-2 \epsilon _{1,m _2} (q_1\cdot \epsilon
_1)\right)\right.\\&+\epsilon _{1,m _1} \left(2 q_{1,m _2} (q_1\cdot \epsilon _1)+t_1 \epsilon _{1,m _2}\right)\\&+\left.2 v_{m
	_1} v_{m _2} \left(q_1\cdot \epsilon _1\right){}^2\right]\,,\\
&\\
A_{m_1 m_2}^{[2]_9\otimes[2]_9\rightarrow[2]_9\rightarrow[2]_4}&= \frac{\alpha '}{4 \sqrt{91}}[ \left(q_{1,m _1} \left(\epsilon _{1,m _2} (q_1\cdot \epsilon _1)+3 q_{1,m _2}\right)\right.\\&+\epsilon
_{1,m _1} \left(q_{1,m _2} (q_1\cdot \epsilon _1)-3 t_1 \epsilon _{1,m _2}\right)\\&+\left.v_{m _1} v_{m _2}
\left(q_1\cdot \epsilon _1\right){}^2\right)]\,,\\
&\\
\eea{spin2part1}
\bea
A_{m_1m_2}^{[1,1,1]_9\otimes[1,1,1]_9\rightarrow[2,2]_9\rightarrow[2]_4}&= \frac{\alpha ' }{2 \sqrt{14}}[\left(q_{1,m _1} \left(q_{1,m _2}-\epsilon _{1,m _2} (q_1\cdot \epsilon _1)\right)\right.\\&+\epsilon _{1,m
	_1} \left(-q_{1,m _2} \left(q_1\cdot \epsilon _1\right)-t_1 \epsilon _{1,m _2}\right)\\&+\left.v_{m _1}
v_{m _2} \left(-\left(q_1\cdot \epsilon _1\right){}^2-t_1\right)\right)]\,,\\
&\\
A_{m_1 m_2}^{[1,1,1]_9\otimes[1,1,1]_9\rightarrow[2]_9\rightarrow[2]_4}&=\frac{\alpha '}{4 \sqrt{21}}[ \left(q_{1,m _1} \left(\epsilon _{1,m _2} (q_1\cdot \epsilon _1)-q_{1,m _2}\right)\right.\\&+\epsilon _{1,m
	_1} \left(q_{1,m _2} (q_1\cdot \epsilon _1)+t_1 \epsilon _{1,m _2}\right)\\&+\left.v_{m _1} v_{m _2}
\left(\left(q_1\cdot \epsilon _1\right){}^2+t_1\right)\right)]\,,\\
&\\
\left( A_{\mathbf{m}}^{[1,1,1]_9\otimes[1,1,1]_9\rightarrow[2,2,2]_9\rightarrow[2]_4}\right)^2 &=\frac{\left(\alpha '\right)^2}{1920} \left(37 t_1 \left(q_2\cdot \epsilon _1\right){}^2-74 t_1
	(\epsilon _1\cdot \epsilon _2) (q_2\cdot \epsilon _1) (q_2\cdot \epsilon _2)\right.\\&\left.+17 \left(q_1\cdot
	\epsilon _1\right){}^2 \left(\left(q_2\cdot \epsilon _2\right){}^2+t_2\right)+17 t_1
	\left(\left(q_2\cdot \epsilon _2\right){}^2+t_2\right)\right.\\&+\left(q_1\cdot \epsilon
	_2\right){}^2 \left(117 \left(q_2\cdot \epsilon _1\right){}^2+37 t_2\right)\\&-74 (q_1\cdot
	\epsilon _1) (q_1\cdot \epsilon _2) \left((q_2\cdot \epsilon _1) (q_2\cdot \epsilon _2)+t_2
	(\epsilon _1\cdot \epsilon _2)\right)\\&+\left(q_1\cdot q_2\right){}^2 \left(117
	\left(\epsilon _1\cdot \epsilon _2\right){}^2-37\right)\\&\left.+q_1\cdot q_2 \left(2 q_1\cdot
	\epsilon _2 \left(37 q_2\cdot \epsilon _2-117 \epsilon _1\cdot \epsilon _2 q_2\cdot
	\epsilon _1\right)\right.\right.\\&\left.\left.+74 q_1\cdot \epsilon _1 \left(q_2\cdot \epsilon _1-\epsilon _1\cdot
	\epsilon _2 q_2\cdot \epsilon _2\right)\right)-37 t_1 t_2 \left(\epsilon _1\cdot
	\epsilon _2\right){}^2\right)\,,
\eea{eq:list_amplitudes_spin2}

and 8-fold degenerate spin 0 states
\bea
8\, \bullet_4\\
A_{[2]_9 \otimes [2]_9 \rightarrow [4]_9 \rightarrow \bullet_4}&=\frac{1}{48} \sqrt{\frac{35}{286}} \alpha ' \left(4 t_1-15 \left(q_1\cdot \epsilon _1\right){}^2\right)\,,\\
&\\
	A_{[2]_9 \otimes [2]_9 \rightarrow [2,2]_9 \rightarrow \bullet_4}&=\frac{1}{24} \sqrt{\frac{5}{7}} \alpha ' \left(3 \left(q_1\cdot \epsilon _1\right){}^2+t_1\right)\,,\\
	&\\
	A_{[2]_9 \otimes [2]_9 \rightarrow [2]_9 \rightarrow \bullet_4}&=\frac{1}{8} \sqrt{\frac{5}{91}} \alpha ' \left(\left(q_1\cdot \epsilon _1\right){}^2-2 t_1\right)\,,\\
	&\\
	A_{[2]_9 \otimes [2]_9 \rightarrow \bullet_9 \rightarrow\bullet_4}&=\frac{\alpha '}{8 \sqrt{11}} \left(\left(q_1\cdot \epsilon _1\right){}^2-t_1\right)\,,\\
	&\\
	A_{[1,1,1]_9 \otimes [1,1,1]_9 \rightarrow [2,2]_9\rightarrow \bullet_4}&=\frac{1}{8} \sqrt{\frac{3}{7}} \alpha ' \left(-\left(q_1\cdot \epsilon _1\right){}^2-t_1\right)\,,\\
	&\\
\eea{scalarspart1}
\bea	
	A_{[1,1,1]_9 \otimes [1,1,1]_9 \rightarrow [2]_9\rightarrow \bullet_4}&=\frac{1}{8} \sqrt{\frac{5}{21}} \alpha ' \left(\left(q_1\cdot \epsilon _1\right){}^2+t_1\right)\,,\\
	&\\
	A_{[1,1,1]_9 \otimes [1,1,1]_9 \rightarrow \bullet_9\rightarrow \bullet_4}&=\frac{\alpha ' }{8 \sqrt{21}}\left(\left(q_1\cdot \epsilon _1\right){}^2+t_1\right)\,,\\
	&\\
	\left( A_{[1,1,1]_9 \otimes [1,1,1]_9 \rightarrow [2,2,2]_9\rightarrow \bullet_4}\right)^2 &=\frac{\left(\alpha '\right)^2}{4480} \left(-49 t_1 \left(q_2\cdot \epsilon _1\right){}^2+98 t_1
	(\epsilon _1\cdot \epsilon _2) (q_2\cdot \epsilon _1) (q_2\cdot \epsilon _2) \right.\\&-49 \left(q_1\cdot
	\epsilon _2\right){}^2 \left(\left(q_2\cdot \epsilon _1\right){}^2+t_2\right)-29
	\left(q_1\cdot \epsilon _1\right){}^2 \left(\left(q_2\cdot \epsilon
	_2\right){}^2+t_2\right)\\&-29 t_1 \left(\left(q_2\cdot \epsilon
	_2\right){}^2+t_2\right)-49
	\left(q_1\cdot q_2\right){}^2 \left(\left(\epsilon _1\cdot \epsilon
	_2\right){}^2-1\right)\\&+98 (q_1\cdot \epsilon _1) (q_1\cdot \epsilon _2) \left((q_2\cdot
	\epsilon _1) (q_2\cdot \epsilon _2)+t_2 (\epsilon _1\cdot \epsilon _2)\right)\\&\left.-98 q_1\cdot q_2 \left(q_1\cdot \epsilon _2 \left(q_2\cdot
	\epsilon _2-(\epsilon _1\cdot \epsilon _2) (q_2\cdot \epsilon _1)\right)\right.\right.\\&\left.\left.+q_1\cdot \epsilon
	_1 \left(q_2\cdot \epsilon _1-(\epsilon _1\cdot \epsilon _2)( q_2\cdot \epsilon
	_2)\right)\right)+49 t_1 t_2 \left(\epsilon _1\cdot \epsilon _2\right){}^2\right)\,. 
\eea{eq:list_amplitudes_scalar}	
Upon contracting each of these amplitudes (with a Pomeron $P_1$) with the appropriate projector and another three-point amplitude (with a Pomeron $P_2$), we recover, upon summing over all 22 states, the level 1 residue of the four-point amplitude, as described in the main text.
