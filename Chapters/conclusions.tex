% Chapter Template

\chapter{Conclusions} % Main chapter title
\label{part:conclusions}

In this thesis, we have discussed two aspects of conformal Regge theory.

In the second chapter, we considered the generalization of the Optical theorem to $ AdS $.
This allowed us to write a formula that relates one loop CFT data in terms of tree level CFT data.

It would be interesting to use this formula to constrain CFT data at one loop in $ AdS $ in various supergravity approximations.
More generally, there are various techniques in the scattering amplitude literature which deal with Feynman diagrams at higher loops which deal with only on-shell data.
It would interesting if they have an analogous construction in $ AdS $.
Since unitarity plays an important role in those constraints, we expect our formula to be useful in that context.

It would be interesting if this formula can be generalized to higher loops.
We commented about it in the conclusion of the second chapter.
It is known that in the supergravity approximation, the loop diagrams can be resummed in the Regge limit.
It would be interesting to generalize it beyond the supergravity approximation.
Recently, there has been a lot of progress in the correlation functions on the boundary of $ AdS_3 $ in terms of Wess-Zumino-Witten models \cite{Eberhardt:2021vsx}.
With or without supersymmetry, these models provide a good arena to test such a resummation procedure in $ AdS $.
It would provide a model of eikonal resummation at finite `t Hooft coupling which can be interpolated between weak and strong coupling.

In particular, cutting rules in flat space suggest that one can always decompose higher loop Feynman diagrams in terms of lower loop diagrams.
Analogously, Witten diagrams might follow such a recursion.

In the third chapter, we considered the generalization of the Regge theory to higher point correlation functions.
As the study of higher point correlation functions is in a nascent stage, there are several avenues of interest.

We have provided a novel basis for three point functions of spinning operators.
This is necesary to establish the orthogonality of the conformal partial waves.
It would be interesting to explicitly write this orthogonality relation and the corresponding Euclidean inversion formula.
Then, one can use the prescription for the Regge limit proposed in the third chapter along with boundedness assumption to arrive at an inversion formula.
This would establish rigidity that is much stronger than the the one imposed by analyticity in spin.

More generally, the study of the causality constraints of the correlation functions has been done only in the four point case.
It would be interesting to generalize this to higher point cases.
Steinmann relations are known to be powerful in the weakly coupled maximally supersymmetric Yang-Mills theory in four dimension (SYM).
It would be interesting to work out their implications in the strongly coupled limit.

Finally, the topic of Regge theory is chiefly inspired by the experimental data.
It would be interesting if the lessons learnt from the higher supersymmetric examples teach us something about the Regge trajectories of the Quantum chromodynamics.
After all, super Yang-Mills theory is a close cousin of the Quantum chromodynamics with a different matter content.
Is there a way to deform the leading Regge trajectory to predict the correspoding leading Regge trajectory of Quantum chromodynamics?
Hopefully, a more detailed understanding of the leading trajectory of conformal Regge theory and SYM will teach some lessons.



\cleardoublepage
